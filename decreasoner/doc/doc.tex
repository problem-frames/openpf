%
% Copyright (c) 2005 IBM Corporation and others.
% All rights reserved. This program and the accompanying materials
% are made available under the terms of the Common Public License v1.0
% which accompanies this distribution, and is available at
% http://www.eclipse.org/legal/cpl-v10.html
%
% Contributors: 
% IBM - Initial implementation
%
\documentclass{article}

\usepackage{url}
\usepackage{moreverb}

\setlength{\topmargin}{-0.35in}
\setlength{\oddsidemargin}{0.25in}
\setlength{\textwidth}{6.0in}
\setlength{\textheight}{8.7in}
\setlength{\parskip}{3pt}
\setlength{\parindent}{0pt}
\renewcommand{\bottomfraction}{1.0}
\renewcommand{\textfraction}{0.10}
\renewcommand{\floatpagefraction}{0.0}
\renewcommand{\topfraction}{1.0}

\begin{document}

\title{Discrete Event Calculus Reasoner \\
Documentation}
\author{Erik T. Mueller \\
IBM Thomas J. Watson Research Center \\
P.O. Box 704 \\
Yorktown Heights, NY 10598 \\
USA}

\bibliographystyle{apalike}
\maketitle
\sloppy

\copyright\ 2005 IBM Corporation and others.

\noindent
All rights reserved. This program and the accompanying materials
are made available under the terms of the Common Public License v1.0
which accompanies this distribution, and is available at
http://www.eclipse.org/legal/cpl-v10.html

\noindent
Contributors: 
IBM - Initial implementation

\section{Introduction}

The Discrete Event Calculus Reasoner is a program for performing
automated commonsense reasoning using the discrete event calculus
\cite{Mueller:2006}, a version of the classical logic event calculus
\cite{Shanahan:1997,MillerShanahan:2002}. The program supports such
types of reasoning as deduction, temporal projection, abduction,
planning, postdiction, and model finding.

The Discrete Event Calculus Reasoner supports the following commonsense
phenomena:
\begin{itemize}
\item
The {\em commonsense law of inertia}, which states that an event
typically changes only a small number of things and everything else in
the world remains unchanged. For example, moving a glass does not
mysteriously cause a glass in another room to move.
\item
{\em Conditional effects of events}. For example, the results of
turning on a television set depend on whether or not it is plugged in.
\item
{\em Release from the commonsense law of inertia}. For example, if a
person is holding a glass, then the location of the glass is released from
the commonsense law of inertia so that the location of the glass is
permitted to vary.
\item
{\em Event ramifications} or indirect effects of events. The tool
supports {\em state constraints}. For example, a glass moves along
with the person holding it. The tool supports {\em causal
constraints}, which deal with the instantaneous propagation of
interacting indirect effects, as in idealized electrical circuits.
\item
{\em Events with nondeterministic effects}. For example, flipping a
coin results in the coin landing either heads or tails.
\item
{\em Gradual change} such as the changing height of a falling object
or volume of a balloon in the process of inflation.
\item
{\em Triggered events} or events that are triggered under certain
conditions. For example, if water is flowing from a faucet into a
sink, then once the water reaches a certain level the water will
overflow.
\item
{\em Concurrent events with cumulative or canceling effects}. For
example, if a shopping cart is simultaneously pulled and pushed, then
it will spin around.
\end{itemize}

The Discrete Event Calculus Reasoner is used as follows.
One places a {\em domain description} into a file. (This file may load
other files.) The domain description consists of
\begin{itemize}
\item
an {\em axiomatization} describing a commonsense domain or domains of
interest,
\item
{\em observations} of world properties at various times, and
\item
a {\em narrative} of known event occurrences.
\end{itemize}
The domain description is expressed using the {\em Discrete Event Calculus
Reasoner language}. One invokes the Discrete Event Calculus Reasoner on the
domain description. The program transforms the domain description into a
{\em satisfiability} (SAT) problem. (The SAT problem is expressed in the
standard format used by most SAT solvers \cite{DIMACS:1993}.) The program
then runs a SAT solver, which produces zero or more solutions, called
{\em models}. The program decodes these models and displays them.

\subsection{Software Requirements}

The Discrete Event Calculus Reasoner runs under Windows and Linux.
To run under Windows, you will first need to install Cygwin, which is
available from
\url{http://www.cygwin.com/}. When installing Cygwin, make sure you
install Python, gcc, and g++. Most Linux installations already have
Python, gcc, and g++. (If you need to get Python, you can get it from
\url{http://www.python.org/}. If you need to get gcc and g++, you can
get them from \url{http://gcc.gnu.org/}.) Under both Windows and
Linux, you will also need to download the following:
\begin{itemize}
\item
One or more SAT solvers. The following are recommended:
Relsat \cite{BayardoSchrag:1997},
which is available from
\url{http://www.almaden.ibm.com/cs/people/bayardo/resources.html},
and
Walksat \cite{SelmanKautzCohen:1993}, which is available from
\url{http://www.cs.washington.edu/homes/kautz/walksat/}.
Walksat can be used to display near miss solutions.
You can also use
Minisat \cite{EenSorensson:2004},
which is available from
\url{http://www.cs.chalmers.se/Cs/Research/FormalMethods/MiniSat/Main.html}.
\item
PLY (Python Lex-Yacc), which is available from
\url{http://www.dabeaz.com/ply/}.
\end{itemize}

\subsection{Installing the Program}

Download the tar file {\tt decreasoner.tar.gz} and
issue the following command:
\begin{verbatim}
tar -zxvf decreasoner.tar.gz
\end{verbatim}
Then continue by following the installation instructions in the file
{\tt decreasoner/README}.

To verify that the program is functioning properly, {\tt cd} to the
{\tt decreasoner} directory and type the following:
\begin{verbatim}
$ python
>>> import decreasoner
>>> decreasoner.test()
\end{verbatim}
If the program is working properly, it will print something like
the following:
\begin{verbatim}
loading examples/Shanahan1997/Yale.e
loading foundations/Root.e
loading foundations/EC.e
28 variables and 64 clauses
relsat solver
1 model
---
model 1:
0
Alive().
Happens(Load(), 0).
1
+Loaded().
Happens(Sneeze(), 1).
2
Happens(Shoot(), 2).
3
-Alive().
-Loaded().
EC: 7 predicates, 0 functions, 0 fluents, 0 events, 0 axioms
Root: 0 predicates, 0 functions, 0 fluents, 0 events, 0 axioms
Yale: 0 predicates, 0 functions, 2 fluents, 3 events, 8 axioms
encoding 0.0s
solution 0.0s
total 0.1s
>>>
\end{verbatim}

\subsection{A Simple Example}

Here is a simple example of how to use the Discrete Event Calculus Reasoner.
We use a text editor to create a file containing the following domain
description:
% ../examples/Manual/Example1.e
\begin{verbatim}
load foundations/Root.e
load foundations/EC.e

sort agent

fluent Awake(agent)
event WakeUp(agent)

[agent,time] Initiates(WakeUp(agent),Awake(agent),time).

agent James
!HoldsAt(Awake(James),0).
Delta: Happens(WakeUp(James),0).

completion Delta Happens

range time 0 1
range offset 1 1
\end{verbatim}
The first line of the file is a comment. The {\tt ;} indicates that it
and the rest of the characters on the line should be ignored.
We always start by loading {\tt Root.e}, which defines the boolean,
integer, predicate, and function sorts, and {\tt EC.e}, which defines
the sorts, functions, predicates, and axioms for the event calculus.

Next we create a simple axiomatization of waking up: We define a new
sort {\tt agent}, to represent a person or animal. We define a new
fluent {\tt Awake} and a new event {\tt WakeUp}.  We add an effect
axiom that states that if an agent wakes up, then that agent will be
awake. Every sentence must be terminated with a period.

Now we code a particular commonsense reasoning problem that uses the
axiomatization. We start by defining a constant {\tt James}, who is an
agent. We state that James is not awake at timepoint 0 and that he
wakes up at timepoint 0. We state that reasoning should be limited to
timepoints 0 through 1.

We save the file as {\tt examples/Manual/Example1.e} and run it by typing
the following:
\begin{verbatim}
$ python
>>> import decreasoner
>>> decreasoner.run('examples/Manual/Example1.e')
\end{verbatim}

The following output is produced:
% ../examples/Manual/Example1.txt
\begin{verbatim}
loading examples/Manual/Example1.e
loading foundations/Root.e
loading foundations/EC.e
6 variables and 10 clauses
relsat solver
1 model
---
model 1:
0
Happens(WakeUp(James), 0).
1
+Awake(James).
P
!Happens(WakeUp(James), 1).
!ReleasedAt(Awake(James), 0).
!ReleasedAt(Awake(James), 1).
EC: 7 predicates, 0 functions, 0 fluents, 0 events, 0 axioms
Example1: 0 predicates, 0 functions, 1 fluents, 1 events, 3 axioms
Root: 0 predicates, 0 functions, 0 fluents, 0 events, 0 axioms
encoding 0.0s
solution 0.0s
total 0.1s
\end{verbatim}
This shows the fluents that are true at timepoint zero, and the
differences in what fluents are true from one timepoint to the next.
Fluents that become true are indicated with a plus sign (``{\tt +}''),
and fluents that become false are indicated with a minus sign
(``{\tt -}'').

An alternative output format is available that shows all the fluents
that are true and false at every timepoint. This format is obtained by
adding the line:
\begin{verbatim}
option timediff off
\end{verbatim}
With this option set, we get the following output:
% ../examples/Manual/Example1a.txt
\begin{verbatim}
loading examples/Manual/Example1a.e
loading foundations/Root.e
loading foundations/EC.e
6 variables and 10 clauses
relsat solver
1 model
---
model 1:
0
!HoldsAt(Awake(James), 0).
!ReleasedAt(Awake(James), 0).
Happens(WakeUp(James), 0).
1
!Happens(WakeUp(James), 1).
!ReleasedAt(Awake(James), 1).
HoldsAt(Awake(James), 1).
P
EC: 7 predicates, 0 functions, 0 fluents, 0 events, 0 axioms
Example1a: 0 predicates, 0 functions, 1 fluents, 1 events, 3 axioms
Root: 0 predicates, 0 functions, 0 fluents, 0 events, 0 axioms
encoding 0.0s
solution 0.0s
total 0.1s
\end{verbatim}

\subsection{More Examples}

The above example showed how the Discrete Event Calculus Reasoner can be
used to perform deduction. Specifically the example demonstrated temporal
projection---given an initial situation and a sequence of events, we
would like to determine the resulting situation. Instead, suppose we
are given an initial situation and a resulting situation, and would
like to reason about what must have happened. This is an example of
abduction. We can perform this type of reasoning by setting up the
commonsense reasoning problem as follows:
% ../examples/Manual/Example2.e
\begin{verbatim}
load foundations/Root.e
load foundations/EC.e

sort agent

fluent Awake(agent)
event WakeUp(agent)

[agent,time] Initiates(WakeUp(agent),Awake(agent),time).

agent James
!HoldsAt(Awake(James),0).
HoldsAt(Awake(James),1).

range time 0 1
range offset 1 1
\end{verbatim}
That is, we assert that James was not awake at timepoint 0 and he was
awake at timepoint 0, but we do not say anything about James waking
up. If we run this, the program abduces that James woke up:
% ../examples/Manual/Example2.txt
\begin{verbatim}
loading examples/Manual/Example2.e
loading foundations/Root.e
loading foundations/EC.e
5 variables and 9 clauses
relsat solver
1 model
---
model 1:
0
Happens(WakeUp(James), 0).
1
+Awake(James).
P
!ReleasedAt(Awake(James), 0).
!ReleasedAt(Awake(James), 1).
EC: 7 predicates, 0 functions, 0 fluents, 0 events, 0 axioms
Example2: 0 predicates, 0 functions, 1 fluents, 1 events, 3 axioms
Root: 0 predicates, 0 functions, 0 fluents, 0 events, 0 axioms
encoding 0.0s
solution 0.0s
total 0.1s
\end{verbatim}

What if we add another agent {\tt Jessie}, but do not specify anything further?
Then we get four models:
% ../examples/Manual/Example3.txt
\begin{verbatim}
loading examples/Manual/Example3.e
loading foundations/Root.e
loading foundations/EC.e
10 variables and 16 clauses
relsat solver
4 models
---
model 1:
0
Happens(WakeUp(James), 0).
Happens(WakeUp(Jessie), 0).
1
+Awake(James).
+Awake(Jessie).
P
!ReleasedAt(Awake(James), 0).
!ReleasedAt(Awake(James), 1).
!ReleasedAt(Awake(Jessie), 0).
!ReleasedAt(Awake(Jessie), 1).
---
model 2:
0
Awake(Jessie).
Happens(WakeUp(James), 0).
Happens(WakeUp(Jessie), 0).
1
+Awake(James).
P
!ReleasedAt(Awake(James), 0).
!ReleasedAt(Awake(James), 1).
!ReleasedAt(Awake(Jessie), 0).
!ReleasedAt(Awake(Jessie), 1).
---
model 3:
0
Awake(Jessie).
Happens(WakeUp(James), 0).
1
+Awake(James).
P
!Happens(WakeUp(Jessie), 0).
!ReleasedAt(Awake(James), 0).
!ReleasedAt(Awake(James), 1).
!ReleasedAt(Awake(Jessie), 0).
!ReleasedAt(Awake(Jessie), 1).
---
model 4:
0
Happens(WakeUp(James), 0).
1
+Awake(James).
P
!Happens(WakeUp(Jessie), 0).
!ReleasedAt(Awake(James), 0).
!ReleasedAt(Awake(James), 1).
!ReleasedAt(Awake(Jessie), 0).
!ReleasedAt(Awake(Jessie), 1).
EC: 7 predicates, 0 functions, 0 fluents, 0 events, 0 axioms
Example3: 0 predicates, 0 functions, 1 fluents, 1 events, 3 axioms
Root: 0 predicates, 0 functions, 0 fluents, 0 events, 0 axioms
encoding 0.0s
solution 0.0s
total 0.1s
\end{verbatim}
The program simply finds all the models consistent with the stated facts and
axioms. (An exception is that occurrences of events at the last 
timepoint are ruled out.)
We haven't said whether Jessie was initially awake or not, and
we haven't said that in order for a person to wake up, the person must
have previously been asleep. If we add Jessie's initial state and
an action precondition axiom for {\tt WakeUp}:
\begin{verbatim}
!HoldsAt(Awake(Jessie),0).
[agent,time] Happens(WakeUp(agent),time) -> !HoldsAt(Awake(agent),time).
\end{verbatim}
then we get only two models:
% ../examples/Manual/Example4.txt
\begin{verbatim}
loading examples/Manual/Example4.e
loading foundations/Root.e
loading foundations/EC.e
10 variables and 19 clauses
relsat solver
2 models
---
model 1:
0
Happens(WakeUp(James), 0).
1
+Awake(James).
P
!Happens(WakeUp(Jessie), 0).
!ReleasedAt(Awake(James), 0).
!ReleasedAt(Awake(James), 1).
!ReleasedAt(Awake(Jessie), 0).
!ReleasedAt(Awake(Jessie), 1).
---
model 2:
0
Happens(WakeUp(James), 0).
Happens(WakeUp(Jessie), 0).
1
+Awake(James).
+Awake(Jessie).
P
!ReleasedAt(Awake(James), 0).
!ReleasedAt(Awake(James), 1).
!ReleasedAt(Awake(Jessie), 0).
!ReleasedAt(Awake(Jessie), 1).
EC: 7 predicates, 0 functions, 0 fluents, 0 events, 0 axioms
Example4: 0 predicates, 0 functions, 1 fluents, 1 events, 5 axioms
Root: 0 predicates, 0 functions, 0 fluents, 0 events, 0 axioms
encoding 0.0s
solution 0.0s
total 0.1s
\end{verbatim}
That is, Jessie might or might not have woken up.

\section{Discrete Event Calculus Reasoner Language}
In this section, we describe the Discrete Event Calculus Reasoner
in more detail.

\subsection{Sorts}
Sentences are expressed in the language of many-sorted first-order predicate
calculus with subsort orders \cite{Walther:1987}. This means that:
\begin{itemize}
\item
sorts can be subsorts of other sorts,
\item
every variable, constant, and function symbol has an associated sort,
\item
every argument position of every function and predicate symbol has an
associated sort, and
\item
for a term to fill an argument position of a function or predicate
symbol, the sort associated with the term must be a subsort of the
sort associated with the argument position.
\end{itemize}

We define a sort called {\tt object} as follows:
\begin{verbatim}
sort object
\end{verbatim}
We define a sort {\tt agent} that is a subsort of {\tt object} as follows:
\begin{verbatim}
sort agent: object
\end{verbatim}

The sort associated with a variable is determined by removing digits
from the variable. For example, the sort of the variable
{\tt snowflake72} is {\tt snowflake}. A constant's sort is specified
when the constant is defined. For example, the following defines
three constants whose sort is {\tt agent}:
\begin{verbatim}
agent Fred, Annie, Thea
\end{verbatim}
Each object in the world is assumed to be named by a unique constant.
That is, the constants {\tt Annie} and {\tt Thea} do not refer to the
same person. This is known as the {\em unique names assumption}.

Here is a definition of a function symbol {\tt Floor} whose sort is
{\tt integer} and whose first and only argument position is of sort
{\tt room}:
\begin{verbatim}
function Floor(room): integer
\end{verbatim}

Here is a definition of a predicate symbol {\tt PartOf} that takes two
arguments of sort {\tt physobj} and {\tt object}:
\begin{verbatim}
predicate PartOf(physobj,object)
\end{verbatim}

\subsection{Formulas}

The following grammar,
which is based on that of the Bliksem theorem prover \cite{deNivelle:1999},
in conjunction with the sort constraints described above, specifies
how sentences are constructed:
\begin{tabbing}
{\em term} ::= \= {\em variable} $\mid$ {\em constant} $\mid$ \\
         \> {\em functionsymbol}{\tt (}{\em arguments}{\tt )} $\mid$ \\
         \> {\em term} {\tt +} {\em term} $\mid$ {\em term} {\tt -} {\em term} $\mid$ {\em term} {\tt *} {\em term} $\mid$ {\em term} {\tt /} {\em term} $\mid$ \\
         \> {\em term} {\tt \%} {\em term} $\mid$ {\tt -} {\em term} $\mid$ \\
         \> {\tt (}{\em term}{\tt )} $\mid$ {\em reifiedformula} \\
{\em formula} ::= \= {\em predicatesymbol}{\tt (}{\em arguments}{\tt )} $\mid$ \\
            \> {\em term} {\tt <} {\em term} $\mid$ {\em term} $\mbox{\tt <=}$ {\em term} $\mid$ {\em term} {\tt =} {\em term} $\mid$ \\
            \> {\em term} $\mbox{\tt >=}$ {\em term} $\mid$ {\em term} {\tt >} {\em term} $\mid$ {\em term} $\mbox{\tt !=}$ {\em term} $\mid$ \\
            \> {\em formula} {\tt |} {\em formula} $\mid$ {\em formula} {\tt \&} {\em formula} $\mid$ {\tt !} {\em formula} $\mid$ \\
            \> {\em formula} $\mbox{\tt ->}$ {\em formula} $\mid$ {\em formula} $\mbox{\tt <->}$ {\em formula} $\mid$ \\
            \> {\tt \{}{\em variables}{\tt \}} {\em formula} $\mid$ {\tt [}{\em variables}{\tt ]} {\em formula} $\mid$ \\
            \> {\tt (}{\em formula}{\tt )} \\
{\em reifiedformula} ::= \= {\em formula} \\
{\em arguments} ::= \= {\em term} $\mid$ {\em arguments}{\tt ,} {\em term} \\
{\em variables} ::= \= {\em variable} $\mid$ {\em variables}{\tt ,} {\em variable}
\end{tabbing}
A {\em variable} consists of one or more lowercase letters followed by
zero or more digits. Examples of variables are {\em agent} and
{\em cat1}.
A {\em constant} consists of (a) one or more digits or (b) an
uppercase letter followed by zero or more letters or digits. Examples
of constants are 63, {\em James}, and {\em Snowball1}.
{\em functionsymbol}s and {\em predicatesymbol}s consist of an
uppercase letter followed by zero or more letters or digits. Examples
of {\em functionsymbol}s are {\em BuildingOf} and {\em SkyOf}; examples of
{\em predicatesymbol}s are {\em PartOf} and {\em Adjacent}.
Fluent and event symbols are predicate symbols.

The meaning of the symbols is as follows:
\begin{center}
\begin{tabular}{|l|l|l|l|} \hline
{\bf Symbol} & {\bf Meaning} & {\bf Symbol} & {\bf Meaning} \\ \hline
{\tt +} & addition &
{\tt !=} & not equal to \\ \hline
{\tt -}  & subtraction, negation &
{\tt |} & disjunction (OR, $\vee$) \\ \hline
{\tt *} & multiplication &
{\tt \&} & conjunction (AND, $\wedge$) \\ \hline
{\tt /} & division &
{\tt !} & logical negation \\ \hline
{\tt \%} & modulus &
{\tt ->} & implication \\ \hline
{\tt <} & less than &
{\tt <->} & bi-implication \\ \hline
{\tt <=} & less than or equal to &
{\tt \{ \}} & existential quantification ($\exists$) \\ \hline
{\tt =} & equal to &
{\tt [ ]} & universal quantification ($\forall$) \\ \hline
{\tt >=} & greater than or equal to &
{\tt ( )} & grouping \\ \hline
{\tt >} & greater than & {\tt ,} & separator \\ \hline
\end{tabular}
\end{center}

\subsection{Completion}
The {\tt completion} statement specifies that a predicate symbol should
be subject to predicate completion. The syntax of this statement is:
\begin{flushleft}
{\tt completion} {\em label} {\em predicatesymbol}
\end{flushleft}

\subsection{Ignore}
The {\tt ignore} statement specifies that one or more predicate symbols
and all sentences containing those predicate symbols should be ignored.
The syntax of this statement is:
\begin{flushleft}
{\tt ignore} {\em predicatesymbol}{\tt ,} {\em predicatesymbol}{\tt ,} \ldots
\end{flushleft}

\subsection{Load}
The {\tt load} statement can be used to load other Discrete Event Calculus
Reasoner language files. The syntax of this statement is:
\begin{flushleft}
{\tt load} {\em filename}
\end{flushleft}

\subsection{Manualrelease}
The {\tt manualrelease} statement inhibits automatic generation of
assertions that particular fluent symbols are not released at time
point 0. The syntax of this statement is:
\begin{flushleft}
{\tt manualrelease} {\em fluentsymbol}{\tt ,} {\em fluentsymbol}{\tt ,} \ldots
\end{flushleft}

\subsection{Mutex and Xor}
The {\tt mutex} statement specifies that one or more event or fluent
symbols are mutually exclusive at each timepoint. The syntax of this
statement is:
\begin{flushleft}
{\tt mutex} {\em symbol}{\tt ,} {\em symbol}{\tt ,} \ldots
\end{flushleft}
Exclusive or (xor) is also provided:
\begin{flushleft}
{\tt xor} {\em symbol}{\tt ,} {\em symbol}{\tt ,} \ldots
\end{flushleft}

\subsection{Noninertial}
The {\tt noninertial} statement specifies that one or more fluent symbols
should not be subject to the commonsense law of inertia, across all
timepoints. The syntax of this statement is:
\begin{flushleft}
{\tt noninertial} {\em fluentsymbol}{\tt ,} {\em fluentsymbol}{\tt ,} \ldots
\end{flushleft}

\subsection{Options}
The {\tt option} statement can be used to specify the values of certain
program options. The syntax of this statement is:
\begin{flushleft}
{\tt option} {\em optionname} {\em optionvalue}
\end{flushleft}
The following {\em optionname}s are supported:
\begin{description}

\item[{\tt debug}]
If the {\em optionvalue} is {\tt on}, detailed debugging output is
produced.
The default value of this option is {\tt off}.

\item[{\tt encoding}]
The value of this option specifies the event calculus-to-SAT encoding
method.  Option 2 is the method described in the FLAIRS
paper \cite{Mueller:2004b}, and option 3 is the slightly different
method described in the {\em Journal of Logic and Computation} article
\cite{Mueller:2004a}. The default value of this option is 2.

\item[{\tt finalstatefile}]
The value of this option specifies the name of a file to which the
final state of (the first model of) the run will be written.

\item[{\tt manualrelease}]
If {\tt on}, automatic generation of assertions of the form
{\tt !Released(fluent,0)} is inhibited, so that such assertions can be
added manually on a case-by-case basis.
This is typically used when pasting together several runs---the final
state of each run, including whether each fluent is {\tt Released}
or not, serves as the initial state of the next run.
The default value of this option is {\tt off}.

\item[{\tt modeldiff}]
If {\tt on}, differences from one model to the next are shown,
instead of complete models. The default value of this option is {\tt off}.

\item[{\tt renaming}]
If {\tt on}, the technique of renaming subformulas
\cite{PlaistedGreenbaum:1986,GiunchigliaSebastiani:1999} is used to
convert to a compact conjunctive normal. The default value of this
option is {\tt on}.

\item[{\tt showpred}]
If {\tt on}, the truth values of all predicates other than fluents and
events are shown. The default value of this option is {\tt off}.

\item[{\tt solver}]
The value of this option specifies the name of the solver to use.
The default value of this option is {\tt relsat}.

\item[{\tt timediff}]
If {\tt on}, differences from one timepoint to the next are shown,
instead of complete states. The default value of this option is {\tt on}.

\item[{\tt tmpfilerm}]
If {\tt on}, temporary files, which are stored in the {\tt /tmp}
directory, are removed at the end of each run. The default value of
this option is {\tt on}.

\item[{\tt trajectory}]
If {\tt on}, {\tt Trajectory} axioms are supported.
The default value of this option is {\tt off}.

%\item[{\tt weighted}]
%If {\tt on}, weighted MAX-SAT is performed. (You must download a MAX-SAT
%solver such as maxwalksat or wmaxsat.)
%The default value of this option is {\tt off}.

\end{description}

\subsection{Ranges}
The {\tt range} statement specifies the range of sorts, such as the
{\tt time} sort, that are subsorts of the {\tt integer} sort.
The syntax is:
\begin{flushleft}
{\tt range} {\em sort} {\em minvalue} {\em maxvalue}
\end{flushleft}

\bibliography{doc}

\end{document}
